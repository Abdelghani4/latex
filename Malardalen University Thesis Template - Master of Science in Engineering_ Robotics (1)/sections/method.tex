\section{Metod (Method)} 
\label{sec:method}

I det här avsnittet ska du beskriva vilka vetenskapliga metoder du har använt och hur du har gått tillväga med själva arbetet. För varje mål ovan identifierar du en metod för att nå målet. Valen av metod ska motiveras. Du kan t ex ha gjort en matematisk modell, använt simuleringar, gjort en implementation som du testat eller gjort experiment som du kanske utvärderat med hjälp av statistiska metoder. Vi avser här i första hand att du beskriver de vetenskapliga metoder du använt, men det är också bra om du ger en beskrivning av hur du arbetat med uppgiften. Avsnittet Metod svarar också på varför du gjorde på ett visst sätt eller varför du använde ett visst verktyg. Du ska alltså inte bara beskriva ”vad” utan  också ”varför”. Ställ dig frågan: kan den valda metoden hjälpa mig att nå de uppsatta målen och därmed besvara frågeställningen?

Att välja rätt vetenskapliga metod(er) är viktigt för att du ska nå dina mål, därför är detta en punkt som du på ett tidigt stadium bör diskutera med din handledare. Sök också i litteraturen efter bra beskrivningar av metoder, och hur du på bästa sätt skriver ett Metodavsnitt. 
