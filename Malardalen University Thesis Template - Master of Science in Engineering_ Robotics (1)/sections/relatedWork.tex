\section{Tidigare arbeten/litteraturöversikt (Related Work)}

Syftet med detta avsnitt är att placera in ditt arbete i ett sammanhang och jämföra det med tidigare publicerade arbeten och resultat inom området. Denna del ska vara grundlig. Du beskriver här existerande kunskap och hur denna utökas av ditt arbete. Den ska innehålla analyser av tidigare arbeten som exempelvis beskriver hur olika metoder skiljer sig åt. Du ska visa på de viktigaste likheterna och skillnaderna beträffande uppgift, angreppssätt/metodologi samt resultat. Det är viktigt att du på ett neutralt sätt diskuterar för- och nackdelar med ditt eget arbete jämfört med andras.

Detta skapar också en förväntan på bidraget för ditt arbete, läsaren lär sig här om begränsningar hos tidigare arbeten och varför din uppgift är en utmaning.. 

Tillsammans kommer detta avsnitt tillsammans med bakgrund att introducera ”state of the art”/”state of practice” och dess brister, betydelsen av uppgiften samt vad ditt arbete ska jämföras med.
