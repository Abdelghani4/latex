\subsection{Nouvelles requêtes}
\subsubsection{Première Approche}
Pour ce nouveau schéma de base de données, les réponses aux requêtes seront maintenant plus facile et moins coûteuses en compétence de calculs. On détaillera ci-dessous les nouvelles requêtes pour les différents cas d'usages. 
\begin{enumerate} 
\item Quelle est la valeur de chaque indicateur pour une région spécifique (par example l’Afrique) ?

Récupérer les données de la table "indicatorsAfrique" qui correspond à la région d'intérêt.
\item Comment obtenir la valeur du Topic ’Poverty’ pour un intervalle d’années pour un pays donné ?

Récupérer les données de la table "indicators" qui correspondent au pays et à l'intervalle d'années d'intérêt.
Filtrer les données en sélectionnant uniquement les indicateurs correspondant au Topic "Poverty".
\item Comment obtenir la manière avec laquelle l’accès à l’électricité a changer pour le groupe ’LowIncome’ ?

Récupérer les données de la table "indicators" qui correspondent aux pays du groupe "LowIncome".
Filtrer les données en sélectionnant uniquement les indicateurs correspondant au Topic "Electricity access".
Calculer les différences ou les tendances entre les années pour voir comment l'accès à l'électricité a changé.
\item Quelles sont les pays africains qui possèdent une profondeur de déficit calorique inférieur au minimum the calories dont une personne a besoin ?

Récupérer les données de la table "indicatorsAfrique" qui correspondent à la région d'intérêt.
Filtrer les données en sélectionnant uniquement les indicateurs correspondant au Topic "Caloric deficiency".
Sélectionner les pays où la valeur de l'indicateur est inférieure à la valeur minimale nécessaire.
\item Quelle est la valeur du Topic ‘GDP’ pour les ‘high income’ groupes ?

Récupérer les données de la table "indicators" qui correspondent aux pays du groupe "HighIncome".
Filtrer les données en sélectionnant uniquement les indicateurs correspondant au Topic "GDP".
Calculer une valeur agrégée, comme la moyenne, pour les données filtrées.
\item Quelle est la croissance du GDP pour les pays de l’OCDE, par rapport à la base de données de l’année 2010 ?\\
Étape 1 : Sélectionner uniquement les enregistrements des pays appartenant à l'OCDE.\\
Étape 2 : Sélectionner les enregistrements correspondant à l'année 2010 pour le "topic" GDP.\\
Étape 3 : Pour chaque pays, calculer la différence entre la valeur de GDP de l'année 2010 et la valeur de base (par exemple, la moyenne de la série historique de GDP).\\
Étape 4 : Tri des pays par ordre croissant ou décroissant de croissance du GDP.\\
\item Quelle est le pourcentage de chômage ‘unemployment’ dans chaque Région ?\\
Étape 1 : Sélectionner les enregistrements pour le "topic" unemployment.\\
Étape 2 : Pour chaque région, calculer la moyenne des valeurs de chômage pour tous les pays de la région.\\
Étape 3 : Tri des régions par ordre croissant ou décroissant de pourcentage de chômage.\\
\item Quelles sont les régions qui ont évolué économiquement de l’année 2005 à l’année 2010 ?\\
Étape 1 : Sélectionner les enregistrements pour l'année 2005 et l'année 2010 pour le "topic" GDP.\\
Étape 2 : Pour chaque pays, calculer la différence entre la valeur de GDP de l'année 2010 et la valeur de l'année 2005.\\
Étape 3 : Pour chaque région, calculer la moyenne de la croissance économique pour tous les pays de la région.\\
Étape 4 : Tri des régions par ordre croissant ou décroissant de croissance économique.\\
\end{enumerate}

\subsubsection{Deuxième Approche}
Pour ce nouveau schéma de base de données, les réponses aux requêtes seront maintenant plus facile et moins coûteuses en compétence de calculs. On détaillera ci-dessous les nouvelles requêtes pour les différents cas d'usages. 

\begin{enumerate} 
    \item Quelle est la valeur de chaque indicateur pour une région spécifique (ex. l’Afrique) ?\\
Utiliser la table "indicators" surchargée avec les attributs "région" et "incomeGroup" pour accéder aux données nécessaires.
Filtrer les données pour la région souhaitée (par exemple l'Afrique).
\item Comment obtenir la valeur du Topic ’Poverty’ pour un intervalle d’années pour un pays donné ?\\
Utiliser la table "indicators" surchargée avec les attributs "région" et "incomeGroup".
Filtrer les données pour le pays souhaité et l'intervalle d'années souhaité.
Utiliser la valeur matérialisée pour le Topic "Poverty".
\item Comment obtenir la manière avec laquelle l’accès à l’électricité a changé pour le groupe ’LowIncome’ ?\\
Utiliser la table "indicatorsLow" pour accéder aux données pour le groupe "LowIncome".
Analysez les données pour l'indicateur "Accès à l'électricité" pour observer les changements.
\item Quelles sont les pays africains qui possèdent une profondeur de déficit calorique inférieur au minimum the calories dont une personne a besoin ?\\
Utiliser la table "indicators" surchargée avec les attributs "région" et "incomeGroup".
Filtrer les données pour la région Afrique.
Sélectionnez les pays avec un niveau de déficit calorique inférieur au minimum nécessaire.
\item Quelle est la valeur du Topic ‘GDP’ pour les ‘high income’ groupes ?\\
Il suffit de faire une requête SQL sur la table "indicatorsHigh" en sélectionnant le Topic "GDP".
\item Quelle est la croissance du GDP pour les pays de l’OCDE, par rapport à la base de données de l’année 2010 ?\\
Il faut d'abord filtrer les pays de l'OCDE en utilisant la table "country" pour sélectionner les pays correspondants.
Ensuite, une requête SQL peut être effectuée sur la table "indicators" pour sélectionner les Topics "GDP" pour ces pays.
Il peut être nécessaire de faire une agrégation sur les valeurs pour obtenir la croissance du GDP en comparaison avec la base de données de l'année 2010.
\item Quelle est le pourcentage de chômage ‘unemployment’ dans chaque Région ?\\
Il faut d'abord sélectionner les valeurs du Topic "unemployment" à partir de la table "indicators".
Ensuite, une requête SQL peut être effectuée sur la table "indicators" en utilisant l'attribut "région" pour grouper les valeurs par région.
Il peut être nécessaire de faire une agrégation pour obtenir le pourcentage de chômage pour chaque région.
\item Quelles sont les régions qui ont évolué économiquement de l’année 2005 à l’année 2010 ?\\
Il faut d'abord sélectionner les valeurs du Topic "GDP" à partir de la table "indicators" pour les années 2005 et 2010.
Ensuite, une requête SQL peut être effectuée sur la table "indicators" en utilisant l'attribut "région" pour grouper les valeurs par région.
Il peut être nécessaire de faire une agrégation pour obtenir la croissance du GDP pour chaque région.
Finalement, en comparant les valeurs pour 2005 et 2010, les régions qui ont évolué économiquement peuvent être identifiées.
\end{enumerate}


\subsection{Nouvelles Statistiques}
\subsubsection{Première approche}

\begin{enumerate}
    \item IndicatorsAfrica.csv
\vspace{2mm}

\textbf{Nombre de documents}: 1,3 millions \\
\textbf{Cardinalité des attributs}

     - 1337 Indicateurs différents 
     
     - 56 années (Year)
     
     - 48 pays (CountryName)

     \item IndicatorsRDM.csv
\vspace{2mm}

\textbf{Nombre de documents}: 4,3 millions \\
\textbf{Cardinalité des attributs}

     - 1344 Indicateurs différents 
     
     - 56 années (Year)
     
     - 199 pays (CountryName)


\vspace{2mm}

    \item Country.csv
\vspace{2mm}

\textbf{Nombre de documents}: 247 \\
\textbf{Cardinalité des attributs}

     - 5 IncomeGroup
     
     - 7 Region
     
     - 247 pays (CountryName)

\vspace{2mm}
    \item SeriesNotes.csv
\vspace{2mm}

\textbf{Nombre de documents}: 369 \\
\textbf{Cardinalité des attributs}

     - 25 Seriescodes différentes
     
     - 55 year


\vspace{2mm}
    \item SeriesCountryNotes.csv
\vspace{2mm}

\textbf{Nombre de documents}: 4857 \\
\textbf{Cardinalité des attributs}

     - 259 IndicatorName

     - 5 AggregationMethod

     - 215 Country Code

     - 37 Topic

\end{enumerate}



 \subsubsection{Deuxième approche}

\begin{enumerate}
    \item IndicatorsHigh.csv
\vspace{2mm}

\textbf{Nombre de documents}: 1,3 millions \\
\textbf{Cardinalité des attributs}

     - 1106 Indicateurs différents 
     
     - 56 années (Year)
     
     - 79 pays (CountryName)

     \item IndicatorsLow.csv
\vspace{2mm}

\textbf{Nombre de documents}: 820000 \\
\textbf{Cardinalité des attributs}

     - 1337 Indicateurs différents 
     
     - 56 années (Year)
     
     - 31 pays (CountryName)

     \item IndicatorsOthers.csv
\vspace{2mm}

\textbf{Nombre de documents}: 3,5 millions \\
\textbf{Cardinalité des attributs}

     - 1344 Indicateurs différents 
     
     - 56 années (Year)
     
     - 137 pays (CountryName)

\vspace{2mm}
    \item Country.csv
\vspace{2mm}

\textbf{Nombre de documents}: 247 \\
\textbf{Cardinalité des attributs}

     - 5 IncomeGroup
     
     - 7 Region
     
     - 247 pays (CountryName)

\vspace{2mm}
    \item SeriesNotes.csv
\vspace{2mm}

\textbf{Nombre de documents}: 369 \\
\textbf{Cardinalité des attributs}

     - 25 Seriescodes différentes
     
     - 55 year

\vspace{2mm}
    \item SeriesCountryNotes.csv
\vspace{2mm}

\textbf{Nombre de documents}: 4857 \\
\textbf{Cardinalité des attributs}

     - 259 IndicatorName

     - 5 AggregationMethod

     - 215 Country Code

     - 37 Topic

\end{enumerate}

